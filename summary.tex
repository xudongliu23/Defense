%\begin{frame}[allowframebreaks]
%\frametitle{References}
%    \bibliographystyle{abbrv}
%    \bibliography{research}
%\end{frame}
%
\frameT{Summary}{
	\begin{enumerate}
		\item The language of P-trees is an intuitive and expressive qualitative preference formalism
					over combinatorial domains, with an edge of lower computational complexity compared to 
					other languages.
		\item With formulas restricted to attributes as node labels, the language of PLP-trees is
					highly accurate in modeling preferences arising in practice, and can be effectively
					learned. Collections of PLP-trees, or PLP-forests, are empirically shown with reduced
					overfitting and higher accuracy.
		\item With a further restriction that all attributes occur exact once on every path,
					the language of LP-trees represent total orders, and thus embrace preference aggregation
					problems using voting schemas (e.g., positional scoring rules).
					As the problems are in general NP-hard, we apply ASP and show ASP tools are effective
					for large instances.
	\end{enumerate}
}

\frameT{Questions?}{
	\begin{center}
		Thank you!
	\end{center}
}
