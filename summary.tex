%\begin{frame}[allowframebreaks]
%\frametitle{References}
%    \bibliographystyle{abbrv}
%    \bibliography{research}
%\end{frame}
%
\frameT{Summary}{
	\begin{enumerate}
%		\item The language of P-trees is an intuitive and expressive qualitative preference formalism
%					over combinatorial domains, with an edge of lower computational complexity compared to 
%					other languages.
%		\item With formulas restricted to attributes as node labels, the language of PLP-trees is
%					highly accurate in modeling preferences arising in practice, and can be effectively
%					learned. Collections of PLP-trees, or PLP-forests, are empirically shown with reduced
%					overfitting and higher accuracy.
%		\item With a further restriction that all attributes occur exact once on every path,
%					the language of LP-trees represent total orders, and thus embrace preference aggregation
%					problems using voting schemas (e.g., positional scoring rules).
%					As the problems are in general NP-hard, we apply ASP and show ASP tools are effective
%					for large instances.
		\item The languages of P-trees and PLP-trees:
		\begin{itemize}
			\item P-trees are expressive with labels being propositional formulas.
			\item PLP-trees are P-trees with labels being attributes.
			\item Both are closely related to existing preference formalisms.
		\end{itemize}
		\item Learning PLP-trees and PLP-forests:
		\begin{itemize}
			\item PLP-trees are highly accurate in modeling preferences arising in practice, 
						and can be effectively learned.
			\item PLP-forests, collections of PLP-trees, are empirically shown with reduced
						overfitting and higher accuracy.
		\end{itemize}
		\item Aggregating LP-trees:
		\begin{itemize}
			\item Preference aggregation problems for LP-trees using positional scoring rules are 
						in general NP-hard.
			\item Answer-set programming tools are effective for large instances.
		\end{itemize}
	\end{enumerate}
}

\frameT{Questions?}{
	\begin{center}
		Thank you!
	\end{center}
}
